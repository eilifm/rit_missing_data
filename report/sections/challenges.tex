%!TEX root = ../paper.tex
\documentclass[../paper.tex]{subfiles}

\begin{document}
\section{Challenges}
The common understanding of OLS model behavior leads that the estimates
of a coefficient for the main effect in our uncorrelated dataset should show no bias.
When using the Drop method, every model fit for every level of missing data and in each replication should, on average, produce unbiased estimates for each of the coefficients.
Assuming that the parameter estimation in the underlying models was low variance,
replication of runs was really just to help reduce the noise. For this purpose, 5 to 10 replications is often more than enough.

\begin{figure}[H]

\centering
\includegraphics[width=.9\textwidth]{drop_invert_bias_demo}
%\caption{Example of a parametric plot ($\sin (x), \cos(x), x$)}
\caption{Plots of responses at various levels of $\beta_{X_{1}X_{2}}$}
\label{fig:drop_invert_bias_demo}
\end{figure}

%ci_rng_demo

%$\textcolor{red}{Add Findings Summary When Complete}


\end{document}

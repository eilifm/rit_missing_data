%!TEX root = ../paper.tex
\documentclass[../paper.tex]{subfiles}

\begin{document}
\section{Abstract}

Extensive research and clear guidelines have emerged for the imputation
of variable sets that exhibit interdependence that can be effectively modeled.
Multiple imputation and its derivative methods serve as the state of the art for
data sets with conforming characteristics. Little guidance can be found
for orthogonal feature sets. Through a factorial experiment, the shape of model
degredation is measured in the presense of various levels of missing data,
data relationship characteristics, and imputation methods.
This paper proposes a series of reccomendations for data imputation on datasets
that show little to no multicolinearity.

\end{document}
% Datasets often have missing data in some or all of the features of interest. Given a data set with independent regressors that can be described using ordinary least squares linear regression, what is the best way to fill in any missing data. There is extensive study on how to leverage feature relationships to fill in missing data however there is little information on methods of independent data sets. The objective of this study is to quantify the impact of various data imputation methods on the model performance and provide recommendations on best practices for independent datasets.

%!TEX root = ../paper.tex
\documentclass[../paper.tex]{subfiles}

\begin{document}
\section{Methodology}
The implementation of the methods explored in this paper were implemented using
the Python/Numpy/Pandas stack with the Statsmodels package providing a robust
implementation of Ordinary Least Squares linear regression.


\subsection{Data Assumptions and Generation Details}
The data set must be comprised of $n$ i.i.d variables. The response $y$ is computed
to follow a linear model with all main effects and two factor interactions
in accordance with equation \ref{eq:data_assumed_model}. In practice, most of
the research focused on the two variable case shown in equation
\ref{eq:2var_assumed_model}.

\begin{equation}\label{eq:data_assumed_model}
  y = \beta_{0} + \beta_{ X_{1} } X_{1}  + \ldots + \beta_{X_{n}}X_{n} +
  \beta_{X_{1}X_{2}}X_{1}X_{2} + \ldots +\beta_{X_{n-1}X_{n}}X_{n-1}X_{n} + \epsilon
\end{equation}


\begin{equation}\label{eq:2var_assumed_model}
  y = \beta_{0} + \beta_{ X_{1}} X_{1} + \beta_{X_{2} }X_{2} +
  \beta_{ X_{1} X_{2} } X_{1}X_{2} + \epsilon
\end{equation}

To simplify data generation, all $\beta$ values are proportional to $\beta_{X_{1}}$.
For example, in a data set with three variables, let $\beta_{X_{1}} = 10$, then array
$[0.1, 1.5]$ defines $\beta_{X_{2}}$ and $\beta_{X_{3}}$ to 1 and 15 respectively.
The $\beta$ values for interaction terms are set explicitly.

As the dataset is artificially generated, noise is added to the final
computation of the response. A \mintinline{Python}{noise_factor} variable is
introduced to control the variance of the noise proportional
to the largest $\beta$ value of the main effects. The mean of the noise must be
zero in accordance with the underlying assumptions of linear regression. These
relationships are described in equations \ref{eq:epsilon_def},
\ref{eq:noise_factor_def} and \ref{eq:noise_factor_sigma_def}.

\begin{equation}\label{eq:epsilon_def}
  \epsilon \textasciitilde N(0, \sigma)
\end{equation}

\begin{equation}\label{eq:noise_factor_def}
  \mintinline{Python}{noise_factor} \geq 0
\end{equation}

\begin{equation}\label{eq:noise_factor_sigma_def}
  \sigma = MAX(\beta_{Main Effects}) * \mintinline{Python}{noise_factor}
\end{equation}




\subsection{Simulating Missing Data}
For a given observation, the probability that a given variable is missing is
idependent from the probability that any other variable is missing. This randomness
assumption is refered to as Missing Completely at Random (MCAR). For the implementation
used in this study, a boolean mask was randomly created for each
of the variables. The selection of data to eliminate can be written as
\mintinline{random_choice(observation indexes, [TRUE, FALSE])} without replacement.
The DataFrame method, \mintinline{Python}{pandas.DataFrame.sample} is used for
performance and simplicity reasons. Removed values were set to NULL. The number of
datapoints to remove is controlled as the fraction of data missing. Given MCAR,
it is possible for more than one variable to be missing. These observations are
removed from the dataset before imputation begins. The response y is never
nulled.


\subsection{Imputation Methods}
%Three methods were used to fill in missing data values
Four methods of data imputations were explored in this experiment.

\begin{enumerate}
  \item Drop Imputation; removal of observations where any value is null.
  \item Mean Imputation; replace missing values of a variable with the mean of
  the available samples for the respective variable.
  \item Random Imputation; random replacement from the set of available samples
  for the the respective variable.
  \item Inversion Imputation; invert values using
\end{itemize}


\begin{equation}
  X_{m} = \begin{bmatrix} \label{eq:lol}
  1 & X_{1} & X_{2} & X_{1}X_{2}
\end{bmatrix}
\end{equation}

The complete observations are given as.
\begin{equation}
  X_{c} =
  \begin{bmatrix} \label{eq:lol}
    X_{m} \neq NULL
  \end{bmatrix}
\end{equation}

Compute the beta estimates for the data using the remaining complete observations.
\begin{equation}
  \hat{\beta}_{c} = (X_{c}^{T}X_{c})^{-1} X_{c}^{T} y
\end{equation}


% \begin{algorithm}
% \caption{Inverse OLS}\label{alg:invert}
% \begin{algorithmic}[1]
% \State $D = $
% \Procedure{Invert}{$a,b$}\Comment{The g.c.d. of a and b}
% \State $X_{n}\gets []$
% \While{$r\not=0$}\Comment{We have the answer if r is 0}
% \State $a\gets b$
% \State $b\gets r$
% \State $r\gets a\bmod b$
% \EndWhile\label{euclidendwhile}
% \State \textbf{return} $b$\Comment{The gcd is b}
% \EndProcedure
% \end{algorithmic}
% \end{algorithm}


\subsection{Experimental Design}

\begin{center}
    \begin{tabular}{ | l | p{1.5in} | p{3in} | }
    \hline
    Variable & Levels & Description \\ \hline
    \mintinline{Python}{sample_size} & 50, 100, 500 &  Lots of stuff here \\ \hline
    \mintinline{Python}{noise_factor} & 0.1, 0.3 &  Lots of stuff here \\ \hline
    \mintinline{Python}{noise_factor} & 0.1, 0.3 &  Lots of stuff here \\ \hline
    \end{tabular}
\end{center}

The objective of the experiment is to understand how

\end{document}

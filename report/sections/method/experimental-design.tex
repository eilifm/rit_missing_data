%!TEX root = ../../paper.tex
\documentclass[../../paper.tex]{subfiles}

\begin{document}
\subsection{Experimental Design}
Originally, a full factorial experiment was constructed to include at least three levels of each to allow for the detection of nonlinear relationships with the response.
Additionally, the full factorial experiment provides resolution for high order interactions. Since the entire experiment is executed on a computer, there were no practical limitation to running all combinations.
All runs are replicated 10 times with the responses averages to ensure representative results.
Each of the imputation methods was performed as its own experiment to allow transformations of the factors and the response if required. The remaining factors and their respective levels can be found in Table \ref{table:factors_and_levels}.

%\begin{table*}[htbp]

\begin{table}[H]
\begin{center}\label{table:factors_and_levels}
    \begin{tabular}{  l | p{1.9in} | l   }
%    \begin{tabular}{ | l | p{1.5in} | p{3in} | }
%      \hline

      \rule{0pt}{14pt} \textbf{Factor} & \textbf{Levels} & \textbf{Description} \\ \hline
      \rule{0pt}{14pt} Imputation Method & Drop, Mean, Inverse, Random & Imputation method used \\ %\hline
      \rule{0pt}{14pt} \# of Observations & 100, 500 &  Samples in the generated data set \\ %\hline
      \rule{0pt}{14pt} $\sigma$ & 0.1, 0.3 &  Relative noise in the response \\ %\hline
      \rule{0pt}{14pt} $\beta_{2}$ Factor & 0.1, 1 &  $\beta_{2} = \beta_{2} Factor * \beta_{1}$ \\ %\hline
      \rule{0pt}{14pt} $\beta_{1,2}$ Factor & 0.000001, 1, 2, 10 &  $\beta_{1,2} = \beta_{1,2} Factor * \beta_{1}$ \\ %\hline
      \rule{0pt}{14pt} Drop Column & $x_{1}$, $x_{2}$ & Variables whose values are destroyed in a run \\ %\hline
    \end{tabular}
    \caption{Factors and Levels}\label{table:factors_and_levels}
\end{center}
\end{table}

Response variables for the experiment were chosen to represent the model performance
and degradation while also being normalized to allow for valid comparison across
models. The three responses closely examined are relative MSE of prediction
with respect to the known data variance $\sigma$ (Relative Prediction MSE), range of the confidence interval for $\hat{\beta}_{i}$ as a Percentage of $\hat{\beta}_{i}$
and percent bias of $\hat{\beta}_{i}$ with respect to the true value of $\beta_{i}$.

%The objective of the experiment is to understand how
Attempts to run the above experiments revealed several issues that prevented direct comparison of the effects within a single model. Among other problems, adjusting the noise level in combination with the coefficient levels introduced compounded signal-to-noise changes that were difficult to standardize into a comparable response.

The above experiments was reduced to two cases which could be discussed independently.
These will be referred to as Case 1 and Case 2 for the remainder of this paper.
They are defined as follows.

\subsubsection{Case 1}
Let $\beta_{ 1} = \beta_{2} = \beta_{1,2} = 10$ such that the underlying model for the data is $y = 10 + 10x_{1} + 10x_{2} + 10x_{1}x_{2} + \epsilon $.
Each dataset included 200 observations.
Since $\beta_{ x_{1}} = \beta_{ x_{2}}$, only one must be simulated to be missing and imputed.
$\epsilon $ remains normally distributed with a mean of zero and two levels of $\sigma$ at 1 and 2. The fraction of missing data ranged from .05 to .7 in increments of .025.

These levels of $\sigma$ yield underlying models with $R^2$ values between .97 and .8.

\subsubsection{Case 2}

Case 2 is identical to Case 1 with one modification. $\beta_{1} = \beta_{1} = 10$ and $\beta_{1,2}$ is set equal to 1. Unfortunately, setting $\beta_{1,2}$ to 1 produces models where $\beta_{1,2}$ is found to be insignificant in the model; this introduces additional anomalies.


%The execution of these two cases proved problemati

\end{document}

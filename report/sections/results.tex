%!TEX root = ../paper.tex
\documentclass[../paper.tex]{subfiles}

\begin{document}

\section{Results and Analysis}
\textcolor{red}{UNDER CONSTRUCTION Pending correction to computation of interactions and rerun of all experiments. The good news is that I did write some of the generic and repeatable plotting code that will be needed regardsless. }
% Inversion imputation forces the missing data onto the line defined by the remaining data. The imputed data set is now biased toward the model with which it was imputed.
% When a model is fit again to the whole data set, the result is a
% tighter CI around the $\beta_{1}$ estimate.
% This bias is magnified as the percentage of missing data increases. The model will
% also converge on zero error as the vast majority of data points will be exactly on the inverted model line. Whether or
% not this is the correct line, the model will converge on perfect fit as seen in the $R^2$ plots.

\begin{figure}[H]

\centering
\includegraphics[width=.9\textwidth]{lol1}
%\caption{Example of a parametric plot ($\sin (x), \cos(x), x$)}
\caption{Plots of responses at various levels of $\beta_{X_{1}X_{2}}$}
\label{fig:lol1}
\end{figure}

\end{document}
